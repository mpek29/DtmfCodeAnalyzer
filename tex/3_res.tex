\section{The Strategy for Solving the Problem}
\subsection{Classic Approach}
\subsubsection{Creating the Orthonormal Basis}
The family of vectors (in this case, our sine functions are vectors belonging to the set of continuous functions on [0,1]) that allows us to create DTMF signals is free (as previously demonstrated). We are constrained to discretize each of these functions for analysis since they are of infinite size.

\lstinputlisting[language=Python]{py/createOrthonormalBasis.py}

\subsubsection{Retrieving the DTMF Code to Analyze}
Next, we retrieve the DTMF code transmission that has been recorded for analysis.

\lstinputlisting[language=Python]{py/getCodeFromFile.py}

\subsubsection{Finding the Linear Combination that Created It}
Then, we will find the linear combination that originated the emitted sound. To do this, we will perform an orthogonal projection of our sound onto the orthonormal basis created previously.

\lstinputlisting[language=Python]{py/projectionOrt.py}

To perform this orthogonal projection, we will compute the inner product of our sound with each vector in the basis.

\lstinputlisting[language=Python]{py/scal.py}

\lstinputlisting[language=Python]{py/integrale.py}

\subsubsection{Identifying the Two Main Frequencies}
To form a DTMF code, two sine waves of different frequencies are used. The goal here is to identify these two frequencies.

\lstinputlisting[language=Python]{py/findingMainFrequencies.py}

\subsubsection{Identifying the Associated Digit}
Each combination of vectors is associated with a telephone digit. The question here is to determine which digit it is.

\lstinputlisting[language=Python]{py/whatNumIsIt.py}
